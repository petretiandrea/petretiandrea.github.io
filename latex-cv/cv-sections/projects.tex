%----------------------------------------------------------------------------------------
%	SECTION TITLE
%----------------------------------------------------------------------------------------

\cvsection{Projects}

%----------------------------------------------------------------------------------------
%	SECTION CONTENT
%----------------------------------------------------------------------------------------

\begin{cventries}

%------------------------------------------------

\cventry
{Creator and maintainer} % Job title
{Home Assistant Tapo Integration} % Organization
{Open Source | Home Automation}
{}
{ % Description(s) of tasks/responsibilities
With 800+ starts, tapo p100 is an integration for controlling smart plugs and smart lights of the Tapo line through the well-known home automation assistant Home Assistant. Made mainly in Python, this is the main integration used in the Home Assistant community.
}

%------------------------------------------------    

\cventry
{Creator and maintainer} % Job title
{Beaesthetic Agenda} % Organization
{Microservices | Backend}
{May 2022 - Now}
{ % Description(s) of tasks/responsibilities
Application and backend for appointment management of a beauty center. The system allows to manage clients, appointments and loyalty cards. It is also able to send notifications via Sms, Whatsapp and in the future push notification to customers to remind them of an appointment.
}

%------------------------------------------------

\cventry
{Contributor} % Job title
{Scanbage} % Organization
{Open Source | AI | University}
{}
{ % Description(s) of tasks/responsibilities
A powerful web app to recognize types of garbage by photo or barcode through convolutional network (CNN Machine Learning). It is a kind of social based on rewards unlocked through the correct differentiation of garbage. The project has been realized in a university context with Gianluca Aguzzi, Marta Luffarelli and Simone Letizi.
}

%------------------------------------------------


\cventry
{Contributor} % Job title
{IntelliSerra} % Organization
{Open Source | University}
{}
{ % Description(s) of tasks/responsibilities
Framework developed in Scala which allows managing smart greenhouse. It allows defining smart greenhouse through sensors and actuators and supports an event-based actuation rules system. The main technologies used in this project are Scala, Akka and Prolog, and it developed with Marta Luffarelli, Simone Letizi and Ylenia Battistini.
}

%------------------------------------------------

\cventry
{Contributor} % Job title
{Fluvium} % Organization
{Open Source | University}
{}
{ % Description(s) of tasks/responsibilities
A full stack system for monitoring river rise levels. The system has been developed starting from embedded components (ESP32) up to the web/cloud layer based on AWS. The project is realized in university context with Gianluca Aguzzi.
}

%------------------------------------------------

\cventry
{Contributor} % Job title
{Face Sketch Recognition - CBIR} % Organization
{AI | Computer Vision | University}
{}
{ % Description(s) of tasks/responsibilities
University project aimed at finding faces based on the similarity of sketches obtained manually or through identikit software.
}

%------------------------------------------------

\end{cventries}